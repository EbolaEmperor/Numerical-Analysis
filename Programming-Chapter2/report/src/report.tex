%!TEX program = pdflatex
\documentclass[11pt,en]{elegantpaper}

\title{The Report for Programming Assignments in Chapter Two}
\author{Wenchong Huang}

\date{\today}

% cmd for this doc
\usepackage{array}
\usepackage{float}
\usepackage{pgfplots}
\usepackage{tikz}
\newcommand{\ccr}[1]{\makecell{{\color{#1}\rule{1cm}{1cm}}}}

\begin{document}

\maketitle


\section{How to Test}

Enter the folder \verb|Programming-Chapter2/src| with terminal, \verb |make| here, you will see some executable files whose names are corresponding assignments. Run them directly and you will see the results.

\section{Results}

\subsection{Assignment B}

Here are the Newton's interpolation results.
\begin{align*}
  p_2(f;x) &= 0.0384615+0.192308\pi_{0}(x)-0.0384615\pi_{1}(x)\\
  p_4(f;x) &= 0.0384615+0.0397878\pi_{0}(x)+0.061008\pi_{1}(x)-0.0265252\pi_{2}(x)+0.00530504\pi_{3}(x)\\
  p_6(f;x) &= 0.0384615+0.0264644\pi_{0}(x)+0.0248454\pi_{1}(x)+0.0149446\pi_{2}(x)-0.0131699\pi_{3}(x)\\
           &  +0.00420316\pi_{4}(x)-0.000840633\pi_{5}(x)\\
  p_8(f;x) &= 0.0384615+0.0223428\pi_{0}(x)+0.013956\pi_{1}(x)+0.0117043\pi_{2}(x)+0.000674338\pi_{3}(x)\\
           &  -0.00489646\pi_{4}(x)+0.00243964\pi_{5}(x)-0.000687223\pi_{6}(x)+0.000137445\pi_{7}(x)
\end{align*}

The following figure shows the images of the interpolating polynomials and the original function.

\tikzset{global scale/.style={
    scale=#1,
    every node/.append style={scale=#1}
  }
}

\begin{center}
  \begin{tikzpicture}[global scale = 0.8]
    \begin{axis}[
        axis lines=middle,
        samples=50,
  %       grid,                
        thick,
        domain=-5:5,
        legend pos=outer north east,
        smooth,
    ]
    \addplot+[color=red!80!black!20,no marks]{0.0384615+0.192308*(\x+5)-0.0384615*(\x+5)*\x};
    \addplot+[color=red!60!black!40,no marks]{0.0384615+0.0397878*(\x+5)+0.061008*(\x+5)*(\x+2.5)-0.0265252*(\x+5)*(\x+2.5)*\x+0.00530504*(\x+5)*(\x+2.5)*\x*(\x-2.5)};
    \addplot+[color=red!40!black!60,no marks]{0.0384615+0.0264644*(\x+5)+0.0248454*(\x+5)*(\x+3.33333)+0.0149446*(\x+5)*(\x+3.33333)*(\x+1.66667)-0.0131699*(\x+5)*(\x+3.33333)*(\x+1.66667)*\x+0.00420316*(\x+5)*(\x+3.33333)*(\x+1.66667)*\x*(\x-1.66667)-0.000840633*(\x+5)*(\x+3.33333)*(\x+1.66667)*\x*(\x-1.66667)*(\x-3.33333)};
    \addplot+[color=red!20!black!80,no marks]{0.0384615+0.0223428*(\x+5)+0.013956*(\x+5)*(\x+3.75)+0.0117043*(\x+5)*(\x+3.75)*(\x+2.5)+0.000674338*(\x+5)*(\x+3.75)*(\x+2.5)*(\x+1.25)-0.00489646*(\x+5)*(\x+3.75)*(\x+2.5)*(\x+1.25)*\x+0.00243964*(\x+5)*(\x+3.75)*(\x+2.5)*(\x+1.25)*\x*(\x-1.25)-0.000687223*(\x+5)*(\x+3.75)*(\x+2.5)*(\x+1.25)*\x*(\x-1.25)*(\x-2.5)+0.000137445*(\x+5)*(\x+3.75)*(\x+2.5)*(\x+1.25)*\x*(\x-1.25)*(\x-2.5)*(\x-3.75)};
    \addplot+[color=black,thick,mark=*]{1/(1+\x*\x)};
    \addlegendentry{$p_2(f;x)$}
    \addlegendentry{$p_4(f;x)$}
    \addlegendentry{$p_6(f;x)$}
    \addlegendentry{$p_8(f;x)$}
    \addlegendentry{$f(x)$}
    \end{axis}
  \end{tikzpicture}
\end{center}

This figure illustrates the Runge phenomenon significantly.

\end{document}
