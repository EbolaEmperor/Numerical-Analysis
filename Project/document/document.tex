%!TEX program = xelatex
% 完整编译: xelatex -> biber/bibtex -> xelatex -> xelatex
\documentclass[lang=cn,11pt,a4paper]{elegantpaper}

\title{设计文档}
\author{Wenchong Huang}
\date{\zhtoday}


% 本文档命令
\usepackage{array}
\newcommand{\ccr}[1]{\makecell{{\color{#1}\rule{1cm}{1cm}}}}

\begin{document}

\maketitle

\section{设计思路}

\begin{itemize}
  \item \verb|class Function|: 函数基类。
  \item \verb|class T_Polynomial<T>|: 多项式模板类。
  \item \verb|class ppForm_base|: 多项式样条插值基类。
  \item \verb|class ppForm_linear|: 线性多项式样条插值类,继承\verb|ppForm_base|。
  \item \verb|class ppForm_cubic|: 三阶多项式样条插值类,继承\verb|ppForm_base|。
  \item \verb|class BSpline_base|: B样条插值基类。
  \item \verb|class BSpline_linear|: 线性B样条插值类,继承\verb|BSpline_base|。
  \item \verb|class BSpline_quadradic|: 二阶B样条插值类,继承\verb|BSpline_base|。
  \item \verb|class BSpline_cubic|: 三阶B样条插值类,继承\verb|BSpline_base|。
  \item \verb|class Curve|: 借助三阶B样条插值实现的曲线生成器。
\end{itemize}

此外,本项目还引用了作者上学期在《优化实用算法》中完成的矩阵库,包括类 \verb|Matrix| 及其派生类 \verb|ColVector|,\verb|RowVector|,封装了矩阵的一些常用运算。篇幅所限,在这篇设计文档中作者不介绍矩阵库的设计思路。

\subsection{class Function}

函数基类规定一个抽象函数,需要具有求值(小括号运算)、求导、求二阶导三项基本功能,其中求值运算定义为虚函数,求导定义小差商为默认方法,允许函数实例用导函数的解析式替换默认方法。原型如下。

\begin{lstlisting}
class Function{
  public:
    virtual double operator () (const double &x) const = 0;
    double diff (const double &x) const;
    double diff2 (const double &x) const;
};
\end{lstlisting}

\subsection{class T\_Polynomial<T>}

多项式模板类。用一个 \verb|std::vector<T>| 存储各项系数,支持多项式的求值、求导、加减乘运算、不同格式的输出。原型如下。

\begin{lstlisting}
template<class T> class T_Polynomial{
  private:
    static int outputMode;

  protected:
    std::vector<T> coef;
    int n;

  public:
    T_Polynomial();
    T_Polynomial(const T &x);
    T_Polynomial(const T &k0, const T &k1);
    T_Polynomial(const std::vector<T> & coef);
    T_Polynomial(const T_Polynomial & p);

    static const int OUTPUT_LATEX = 0;
    static const int OUTPUT_TIKZ = 1;
    static void setOutput(const int & style);
    friend std::ostream & operator << (std::ostream &out, const T_Polynomial &ni);
    
    T operator () (const T &vx) const;
    T_Polynomial operator + (const T_Polynomial &rhs);
    T_Polynomial operator - (const T_Polynomial &rhs);
    T_Polynomial operator * (const T &rhs);
    T_Polynomial operator * (const T_Polynomial &rhs);
    T_Polynomial diff();
};
\end{lstlisting}

定义 \verb|Polynomial| 为模板类的生成类 \verb|T_Polynomial<double>|。

\begin{lstlisting}
typedef T_Polynomial<double> Polynomial;
\end{lstlisting}

\subsection{class ppForm\_base}

主要用于存储分段插值多项式,保存插值结点、每一段的多项式。实现求值(小括号运算)。原型如下。

\begin{lstlisting}
class ppForm_base{
  protected:
      int n;
      std::vector<double> knots;
      std::vector<Polynomial> poly;
  public:
      ppForm_base();
      ppForm_base(const ppForm_base &rhs);
      double operator () (const double &x) const;
};
\end{lstlisting}

\subsection{class ppForm\_linear}

继承\verb|ppForm_base|,实现“给定插值点和对应值”的基本构造函数,调用基本构造函数实现“给定插值点和被拟合函数”、“给定区间、插值点数量、被拟合函数”的构造函数。原型如下。

\begin{lstlisting}
class ppForm_linear : public ppForm_base{
  public:
    ppForm_linear(const std::vector<double> &x, const std::vector<double> &f);
    ppForm_linear(const std::vector<double> &x, Function & func);
    ppForm_linear(const int &n, const double &l, const double &r, Function &func);
};
\end{lstlisting}

\subsection{class ppForm\_cubic}

继承\verb|ppForm_base|,实现“给定插值点和对应值、边界条件”的基本构造函数,调用基本构造函数实现“给定插值点和被拟合函数、边界条件”、“给定区间、插值点数量、被拟合函数、边界条件”的构造函数,同时对三种构造函数给出边界条件缺省时的构造函数,默认边界条件为\textit{natural}。原型如下。

\begin{lstlisting}
class ppForm_cubic : public ppForm_base{
  private:
    // 初始化端点处的函数值条件、端点处的一阶导数与二阶导数连续条件
    void basic_init(Matrix &A, ColVector &b, const std::vector<double> &x, const std::vector<double> &f);

  public:
    // “给定插值点和对应值、边界条件”的基本构造函数。调用basic_init初始化基本条件,再根据bondary添加边界条件,然后求解
    ppForm_cubic(const std::vector<double> &x, const std::vector<double> &f, const std::string &bondary);
    ppForm_cubic(const std::vector<double> &x, const std::vector<double> &f): ppForm_cubic(x, f, "natural") {}

    // “给定插值点和被拟合函数、边界条件”的构造函数
    ppForm_cubic(const std::vector<double> &x, Function &f, const std::string &bondary);
    ppForm_cubic(const std::vector<double> &x, Function &f): ppForm_cubic(x, f, "natural") {}

    // “给定区间、插值点数量、被拟合函数、边界条件”的构造函数
    ppForm_cubic(const int &n, const double &l, const double &r, Function & func, const std::string &bondary);
    ppForm_cubic(const int &n, const double &l, const double &r, Function & func): ppForm_cubic(n, l, r, func, "natural") {}
};
\end{lstlisting}

下面给出一个三阶多项式样条插值的使用实例。以拟合螺旋线的x坐标为例,边界条件为\textit{complete}。以点值格式输出。

\begin{lstlisting}
class Helix_X : public Function{
  double operator () (const double &x) const{
      return x*cos(x);
  }
  double diff (const double &x) const{
      return cos(x)-x*sin(x);
  }
  virtual double diff2 (const double &x) const{
      return -2*sin(x)-x*cos(x);
  }
} hx;
const double pi = acos(-1);
ppForm_cubic pp_x(17, 0, 4*pi, hx, "complete");
cout << "x1 = [";
for(double t = 0; t <= 4*pi; t += 0.01)
  cout << pp_x(t) << " ";
\end{lstlisting}

\subsection{class BSpline\_base}

B样条插值基类。用两个 \verb|std::vector<double>| 分别存储插值点和每个B样条基对应的系数。实现递归求B样条基$B_i^k(x)$,用公式

\begin{equation*}
  \frac{\text{d}}{\text{d} x} B_i^k(x) = \frac{k B_i^{k-1}(x)}{t_{i+k+1} - t_{i-1}} - \frac{k B_{i+1}^{n-1}(x)}{t_{i+k} - t_i}
\end{equation*}

实现B样条基的求导、二阶导。另外定义虚函数$B_i(x)$和对应的导数、二阶导,在线性插值类中该虚函数返回$B_i^1(x)$,在三阶插值类中该虚函数返回$B_i^3(x)$。

最后还要实现求点值(小括号运算)。

\begin{lstlisting}
class BSpline_base{
  protected:
    vector<double> coef;
    vector<double> knots;
    
    double B(const int &i, const int &k, const double &x) const;
    double dB(const int &i, const int &k, const double &x) const;
    double d2B(const int &i, const int &k, const double &x) const;

    virtual double B(const int &i, const double &x) const = 0;
    virtual double dB(const int &i, const double &x) const = 0;
    virtual double d2B(const int &i, const double &x) const = 0;
  
  public:
    BSpline_base(){}
    BSpline_base(const BSpline_base & rhs);
    double operator () (const double &x) const;
  };
\end{lstlisting}

最后,\verb|class BSpline_linear| 与 \verb|class BSpline_cubic| 提供的接口与对应的 ppForm 插值类完全一致,其设计思路不再赘述。

\subsection{class Curve}

给定平面上$n$个点$P_1,...,P_n$,生成一条曲线$\gamma$,使得$\gamma\left(\frac{i-1}{n-1}\right)=P_i\;(i=1,\cdots,n)$。

类的私有成员存储点的数量$n$,以及横坐标与纵坐标的样条曲线$f_x(t),f_y(t)$。构造函数用两个 \verb|vector<double>| 给出点$P_1,...,P_n$的坐标,支持\textit{natural}与\textit{periodic}两种边界条件。支持求值运算(小括号运算)与输出流。原型如下。

\begin{lstlisting}
class Curve{
  private:
    int n;
    BSpline_cubic fx, fy;
  
  public:
    Curve();
    Curve(const vector<double> &x, const vector<double> &y, const std::string & bondary);
    Point operator () (const double &t);
    friend std::ostream & operator << (std::ostream & out, const Curve &rhs);
};
\end{lstlisting}

\end{document}
