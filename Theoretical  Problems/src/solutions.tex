\documentclass[11pt]{elegantbook}

\title{Theoretical Problems}
\subtitle{Numerical analysis 2022}

\author{Wenchong Huang (EbolaEmperor)}
\institute{School of Mathematical Science, Zhejiang University}
\date{September 20th, 2022}

\extrainfo{ Elegantly learning. }

\logo{logo-blue.png}
\cover{cover.jpg}

% modify the color in the middle of titlepage
\definecolor{customcolor}{RGB}{32,178,170}
\colorlet{coverlinecolor}{customcolor}

\begin{document}

\maketitle

\frontmatter
% \tableofcontents

\mainmatter

\chapter{Solving Nonlinear Equations}

\begin{problem}
  Consider the bisection method starting with the initial interval $[1.5,3.5]$. In the following questions "the interval" refers to the bisection interval whose width changes across different loops.
  \begin{itemize}
    \item What is the width of the interval at the $n$th step?
    \item What is the maximum possible distance between the root $r$ and the midpoint of the interval?
  \end{itemize}
\end{problem}

\begin{solution}
  Note that the interval's width is multipled by $\frac{1}{2}$ at each step, and the initial width is $2$, hence the width \textbf{after} the $n$th step is $\frac{1}{2^{n-1}}$.
  
  The maximum distance is not grater than $1$ obviously.

  Since the loop terminated when $|f(c)|<\varepsilon$, we could construct an increasing function $f$ whose root is $1.5+\delta$, and $|f(x)|<\varepsilon$ everywhere, hence the bisection loop will terminate at first step, the distance between midpoint and root is $1-\delta$. Let $\delta\to 0^+$, we know the distance could be infynitely close to $1$.
\end{solution}

\vspace{1.5em}

\begin{problem}
  In using the bisection algorithm with its initial interval as $[a_0,b_0]$ with $a_0>0$, we want to determine the root with its relative error no grater than $\varepsilon$. Prove that this goal of accuracy is guaranteed by the following choice of the number of steps,
  \begin{equation*}
    n\geq\frac{\log(b_0-a_0)-\log\varepsilon-\log a_0}{\log 2}-1
  \end{equation*}
\end{problem}

\begin{solution}
  Suppose the root is $r\geq a_0$. The relative error \textbf{after} the $n$th step is
  \begin{equation}
    \frac{|r-c_n|}{|r|}
  \end{equation}
  
  The following inequations hold
  \begin{equation}
    \frac{|r-c_n|}{|r|} \leq \frac{\frac{1}{2}(b_n-a_n)}{r} \leq \frac{\frac{1}{2}(b_n-a_n)}{a_0} = \frac{b_0-a_0}{a_0 2^{n+1}}
  \end{equation}

  Hence when (1.1) holds, we have
  \begin{align*}
   & (n+1)\log 2 \geq \log(b_0-a_0) -\log\varepsilon -\log a_0\\
  \implies & \log 2^{n+1} \geq \log \left(\frac{b_0-a_0}{\varepsilon a_0}\right)\\
  \implies & 2^{n+1} \geq \frac{b_0-a_0}{\varepsilon a_0} \implies \frac{b_0-a_0}{a_0 2^{n+1}} \leq \varepsilon
  \end{align*}

  Hence the conclution is proved by (1.2).
\end{solution}

\vspace{1.5em}

\begin{problem}
  Perform four iterations of Newton's method for the polynomial equation $p(x)=4x^3-2x^2+3=0$ with the starting point $x_0=-1$. Use a hand calculator and organize results of the iterations in a table.
\end{problem}

\begin{solution}
Firstly we derivate $p(x)$
\begin{equation*}
  p'(x)=12x^2-4x
\end{equation*}

The results are shown as the following table.
  \begin{table}[htbp]
    \centering
    \small
    \begin{tabular}{|c|c|c|c|c|}
    \hline
    \textbf{$n$} & \textbf{$x_n$} & \textbf{$p(x_n)$} & \textbf{$p'(x_n)$} & \textbf{$x_n-\frac{f(x_n)}{f'(x_n)}$} \\ \hline
    \textbf{0}   & -1             & -3                & 16                 & -0.8125                               \\ \hline
    \textbf{1}   & -0.8125        & -0.46582          & 11.1719            & -0.770804                             \\ \hline
    \textbf{2}   & -0.770804      & -0.0201359        & 10.2129            & -0.768832                             \\ \hline
    \textbf{3}   & -0.768832      & -3.98011e-05      & 10.1686            & -0.768828                             \\ \hline
    \textbf{4}   & -0.768828      &                   &                    &                                       \\ \hline
    \end{tabular}
  \end{table}
\end{solution}

\begin{problem}
  Consider a variation of Newton's method in which only the derivative at $x_0$ is used,
  \begin{equation}
    x_{n+1}=x_n-\frac{f(x_n)}{f'(x_0)}
  \end{equation}

  Find $C$ and $s$ such that
  \begin{equation*}
    e_{n+1}=Ce_n^s
  \end{equation*}

  where $e_n$ is the error of Newton's method at step $n$, $s$ is a constant, and $C$ may depend on $x_n$, the given function $f$ and its derivatives.
\end{problem}

\begin{solution}
  Assume the root is $r$, then $e_n=x_n-r$. Let $g(x)=f(r+x)$. By (1.3), we derive 
  \begin{equation*}
    e_{n+1}=e_n-\frac{g(e_n)}{g'(e_0)}=\left(1-\frac{g(e_n)}{e_n g'(e_0)}\right)e_n
  \end{equation*}

  Let $C(n)=1-\frac{g(e_n)}{e_n g'(e_0)}$ and $s=1$, we got $e_{n+1}=C(n)e_n$, and 
  \begin{equation*}
    \lim_{n\to \infty} C(n) = 1-\frac{g'(0)}{g'(e_0)}= 1-\frac{f'(r)}{f'(x_0)}
  \end{equation*}
\end{solution}

\vspace{1.5em}

\begin{problem}
  Within $\left(-\frac{\pi}{2},\frac{\pi}{2}\right)$, will the iteration $x_{n+1}=\tan^{-1} x_n$ converge?
\end{problem}
\begin{solution}
  As we all know that $0<\tan^{-1} x < x \;(x>0)$, so if $x_0>0$, we derive
  \begin{equation*}
    0<x_{n+1}=\tan^{-1} x_n < x_n
  \end{equation*}

  And sequence $\{x_n\}$ has lower bound $0$, so $\{x_n\}$ is convergent by monotinic sequence theorem.

  For $x_0<0$, $\{-x_n\}$ is convergent by the discussion above, hence $\{x_n\}$ is convergent.

  For $x_0=0$, clearly $x_n=0\;(\forall n)$.
\end{solution}

\vspace{1.5em}

\begin{problem}
  Let $p>1$. What is the value of the following continued fraction?
  \begin{equation*}
    x=\frac{1}{p+\frac{1}{p+\frac{1}{p+\cdots}}}
  \end{equation*}

  Prove that the sequence of values converges.
\end{problem}

\begin{solution}
  We construct a sequence by $x_1=\frac{1}{p}$ and $x_{n+1}=\frac{1}{p+x_n}\;(n\geq 1)$, then $x=\lim\limits_{n\to\infty} x_n$ if it exists.

  Consider function $g(x)=\frac{1}{p+x}$, clearly $g(x)\in[0,1]$ for all $x\in[0,1]$. And
  \begin{equation*}
    \lambda = \max_{x\in[0,1]} |g'(x)| = \max_{x\in[0,1]} -\frac{1}{(x+p)^2} = \frac{1}{p^2} < 1
  \end{equation*}

  Hence $g$ is a contraction in $[0,1]$, and consider equation
  \begin{equation*}
    x=g(x)=\frac{1}{p+x}
  \end{equation*}

  the roots are $\frac{-p\pm\sqrt{p^2+4}}{2}$, hence $g$ has unique fixed-point $\alpha=\frac{-p+\sqrt{p^2+4}}{2}$ in $[0,1]$.

  Recall that $x_1=\frac{1}{p}\in[0,1]$, and $x_{n+1}=g(x_n)$. By Theorem 1.38, $\{x_n\}$ converges and $x=\lim\limits_{n\to\infty} x_n=\alpha$. 
\end{solution}

\vspace{1.5em}

\begin{problem}
  What happens in problem 1.2 if $a_0<0<b_0$? Derive an inequality of the number of steps similar to that in problem 1.2. In this case, is the relative error still an appropriate measure?
\end{problem}

\begin{solution}
  In this problem we let the absolutely error $|r-c_n|<\delta$, we derive
  \begin{equation}
    |r-c_n| \leq \frac{1}{2}(b_n-a_n) = \frac{b_0-a_0}{2^{n+1}}
  \end{equation}

  It is sufficient to let $\frac{b_0-a_0}{2^{n+1}}<\delta$, hence $n\geq \frac{\log(b_0-a_0)-\log \delta}{\log 2}-1$.

  We can't use relative error since $r$ might be zero.
\end{solution}

\chapter{Polynomial Interpolation}

\begin{problem}
  For $f\in\mathcal{C}^2[x_0,x_1]$ and $x\in(x_0,x_1)$, linear interpolation of $f$ at $x_0$ and $x_1$ yields
  \begin{equation}
    f(x)-p_1(f;x) = \frac{f''(\xi(x))}{2}(x-x_0)(x-x_1)
  \end{equation}

  Consider the case $f(x)=\frac{1}{x},\;x_0=1,\;x_1=2$.
  \begin{itemize}
    \item Determine $\xi(x)$ explicity.
    \item Extend the domain of $\xi$ continuously from $(x_0,x_1)$ to $[x_0,x_1]$. Find $\max \xi(x)$, $\min \xi(x)$ and
    
    $\max f''(\xi(x))$.
  \end{itemize}
\end{problem}

\begin{solution}
  \begin{enumerate}
    \item The Lagrange's formula yields
    \begin{equation*}
      p_1(f;x) = \frac{(x-2)}{(1-2)} + \frac{1}{2}\times \frac{(x-1)}{(2-1)} = -\frac{1}{2}x+\frac{3}{2}
    \end{equation*}
    Substitute it to (2.1), with $f''(x)=2x^{-3}$, yield
    \begin{equation*}
      \frac{1}{x}+\frac{1}{2}x-\frac{3}{2} = (x-1)(x-2)\xi^{-3}(x)
    \end{equation*}
    The result follows from it:
    \begin{equation*}
      \xi(x)=\sqrt[3]{2x}
    \end{equation*}

    \item $\xi(x)$ is increasing in $[1,2]$, hence
    \begin{equation*}
      \max \xi(x)=\xi(2)=\sqrt[3]{4},\qquad \min \xi(x)=\xi(1) = \sqrt[3]{2}
    \end{equation*}
    Also
    \begin{equation*}
      f''(\xi(x))=2\left(\sqrt[3]{2x}\right)^{-3}=\frac{1}{x}
    \end{equation*}
    is decreasing in $[1,2]$, hence
    \begin{equation*}
      \max f''(\xi(x)) = f''(\xi(1)) = 1
    \end{equation*}
  \end{enumerate}
\end{solution}

\vspace{1.5em}

\begin{problem}
  Let $\mathbb{P}_m^+$ be the set of all polynomials of degree $\leq m$ that are non-negative on the real line,
  \begin{equation*}
    \mathbb{P}_m^+ = \{p:p\in \mathbb{P}_m,\; \forall x\in \mathbb{R}, \; p(x)\geq 0\}
  \end{equation*}
  Find $p\in \mathbb{P}_{2n}^+$ such that $p(x_i)=f_i$ for $i=0,1,...,n$ where $f_i\geq 0$ and $x_i$ are distinct points on $\mathbb{R}$.
\end{problem}

\begin{solution}
  Let $q(x)\in\mathbb{P}_n$ be the unique interpolation polynomial satisfies
  \begin{equation*}
    q(x_i)=\sqrt{f_i},\qquad i=0,1,...,n
  \end{equation*}
  Let $p(x)=q^2(x)$, then $p(x)\in \mathbb{P}_{2n}^+$ and
  \begin{equation*}
    p(x_i)=q^2(x_i)=f_i,\qquad i=0,1,...,n
  \end{equation*}
  Hence $p(x)$ is what we need. The Lagrange's interpolation formula yields:
  \begin{equation*}
    p(x)=\left(\sum_{i=0}^n \sqrt{f_i}\prod_{j=0,j\neq i}^n \frac{x-x_j}{x_i-x_j}\right)^2
  \end{equation*}
\end{solution}

\vspace{1.5em}

\begin{problem}
  Cnosider $f(x)=e^x$.
  \begin{itemize}
    \item Prove by induction that
    \begin{equation}
      \forall t\in \mathbb{R}, \qquad f[t,t+1,...,t+n]=\frac{(e-1)^n}{n!}e^t
    \end{equation}
    \item From Corollary 2.22 we know
    \begin{equation}
      \exists \xi\in(0,n) \; \text{s.t.} \; f[0,1,...,n]=\frac{1}{n!}f^{(n)}(\xi)
    \end{equation}
    Determine $\xi$ from the above two equations. Is $\xi$ located to the left or to the right of the midpoint $n/2$.
  \end{itemize}
\end{problem}

\begin{solution}
  \begin{enumerate}
    \item The Lagrange's formuula yields
    \begin{equation*}
      p(f;x) = \sum_{k=0}^n e^{t+k} \prod_{j=0,j\neq k}^n \frac{x-x_j}{x_k-x_j} = e^t\sum_{k=0}^n e^{k} \frac{(-1)^{n-k}\prod_{j=0,j\neq k}^nx-x_j}{k!(n-k)!}
    \end{equation*}
    Hence
    \begin{equation*}
      f[t,t+1,...,t+n] = e^t\sum_{k=0}^n \frac{(-1)^{n-k}e^k}{k!(n-k)!} = \frac{e^t}{n!}\sum_{k=0}^n \binom{n}{k}(-1)^{n-k}e^k = \frac{(e-1)^n}{n!}e^t
    \end{equation*}

    \item Let $t=0$ in (2.2) and yield
    \begin{equation*}
      f[0,1,...,n]=\frac{(e-1)^n}{n!}
    \end{equation*}
    Substitute it to (2.3), with $f^{(n)}(x)=e^x$, yield
    \begin{equation*}
      \frac{(e-1)^n}{n!} = \frac{e^\xi}{n!}
    \end{equation*}
    The result follows from it:
    \begin{equation*}
      \xi=n\ln(e-1)>\frac{n}{2}
    \end{equation*}
    Hence $\xi$ is located to the right of the midpoint.
  \end{enumerate}
\end{solution}

\vspace{1.5em}

\begin{problem}
  Consider $f(0)=5,\; f(1)=3,\; f(3)=5,\; f(4)=12$.
  \begin{itemize}
    \item Use the Newton's formula to obtain $p_3(f;x)$;
    \item The data suggests that $f$ has a minimum in $x\in(1,3)$. Find an approximate value for the location $x_\text{min}$ of the minimum.
  \end{itemize}
\end{problem}

\begin{solution}
  \begin{enumerate}
    \item The result follows from Newton's interpolation formula:
    \begin{equation*}
      p_3(f;x)=5-2x+x(x-1)+\frac{1}{4}x(x-1)(x-3)
    \end{equation*}
    Transform it into the canonical form:
    \begin{equation*}
      p_3(f;x)=\frac{1}{4}x^{3}-\frac{9}{4}x + 5
    \end{equation*}
    \item Firstly, calculate the derivative of $p_3(f;x)$:
    \begin{equation*}
      p'_3(f;x)=\frac{3}{4}x^{2}-\frac{9}{4}
    \end{equation*}
    The first-order necessary condition $p'_3(f;x)=0$ yields that
    \begin{equation*}
      x_\text{extreame} = \pm \sqrt{3}
    \end{equation*}
    In $x\in(1,3)$, the extreame point might be $x^*=\sqrt{3}$. The second-order condition shows that
    \begin{equation*}
      p''_3(f;x^*)=\frac{3}{2}x^* = \frac{3\sqrt{3}}{2} > 0
    \end{equation*}
    Hence $x^*$ is the minimum, and $x_\text{min}=\sqrt{3}\approx 1.73205$.
  \end{enumerate}
\end{solution}

\vspace{1.5em}

\begin{problem}
  Consider $f(x)=x^7$.
  \begin{itemize}
    \item Compute $f[0,1,1,1,2,2]$.
    \item We konw that this devided difference is expressible in terms of the $5$th derivative of $f$ evaluated at some $\xi\in(0,2)$. Determine $\xi$.
  \end{itemize}
\end{problem}

\begin{solution}
  \begin{enumerate}
    \item Solve the Hermite's interpolation with a difference table. The result of Newton's form follows:
    \begin{equation*}
      p(x)=x+6x(x-1)+15x(x-1)^2+42x(x-1)^3+30x(x-1)^3(x-2)
    \end{equation*}
    Hence
    \begin{equation*}
      f[0,1,1,1,2,2]=30
    \end{equation*}

    \item The $5$th derivate of $f$ is
    \begin{equation*}
      f^{(5)}(x)=2520x^2
    \end{equation*}
    Then $\frac{f^{(5)}(x)}{5!}=f[0,1,1,1,2,2]$ yields
    \begin{equation*}
      \frac{2520}{5!}\xi^2=30 \qquad \implies \qquad \xi=\sqrt{\frac{10}{7}}\approx 1.42857\in(0,2)
    \end{equation*}
  \end{enumerate}
\end{solution}

\vspace{1.5em}

\begin{problem}
  $f$ is a function on $[0,3]$ for which one knows that
  \begin{equation*}
    f(0)=1,\quad f(1)=2,\quad f'(1)=-1,\quad f(3)=f'(3)=0
  \end{equation*}
  \begin{itemize}
    \item Estimate $f(2)$ using Hermite's interpolation.
    \item Estimate the maximum possible error of the above answer if one konws, in addition, that $f\in \mathcal{C}^5[0,3]$ and $|f^{(5)}(x)|\leq M$ on $[0,3]$. Express the answer in terms of $M$.
  \end{itemize}
\end{problem}

\begin{solution}
  \begin{enumerate}
    \item The Hermite's interpolation gives that
    \begin{equation*}
      p(x)=1+x-2x(x-1)+\frac{2}{3}x(x-1)^2-\frac{5}{36}x(x-1)^2(x-3)
    \end{equation*}
    Hence, estimate $f(2)$ as
    \begin{equation*}
      f(2)\approx p(2)= \frac{11}{18} \approx 0.611111
    \end{equation*}

    \item Theorem 2.35 gives that
    \begin{equation*}
      f(x)-p(x)=\frac{f^{(5)}(\xi)}{120}x(x-1)^2(x-3)^2
    \end{equation*}
    The result follows directly:
    \begin{equation*}
      |f(2)-p(2)|=\left|\frac{f^{(5)}(\xi)}{60}\right| \leq \frac{M}{60}
    \end{equation*}
  \end{enumerate}
\end{solution}

\vspace{1.5em}

\begin{problem}
  Define foward difference by
  \begin{align*}
    \Delta f(x) = f(x+h) - f(x) ,\qquad \Delta^{k+1} f(x) = \Delta \Delta^k f(x) = \Delta^k f(x+h) - \Delta^k f(x)
  \end{align*}
  and backward difference by
  \begin{align*}
    \nabla f(x) = f(x) - f(x-h), \qquad \nabla^{k+1} f(x) = \nabla \nabla^k f(x) = \nabla^k f(x) - \nabla^k f(x-h)
  \end{align*}
  Prove
  \begin{align}
    \Delta^k f(x) &= k!h^kf[x_0,x_1,...,x_k]\\
    \nabla^k f(x) &= k!h^kf[x_0,x_{-1},...,x_{-k}]
  \end{align}
  where $x_j=x+jh$.
\end{problem}

\begin{solution}
 The Lagrange's interpolation formula yields
 \begin{equation*}
  f[x_0,x_1,...,x_k] = \sum_{i=0}^k f(x_i) \frac{1}{\prod_{j=1,j\neq i}^k (x_i-x_j)} = \sum_{i=0}^k \frac{(-1)^{k-i}f(x+ih)}{h^ki!(k-i)!}
 \end{equation*}
 It yields an equivalent form of (2.4):
 \begin{equation}
  \Delta^k f(x)=k!h^kf[x_0,x_1,...,x_k]=\sum_{i=0}^k \binom{k}{i}(-1)^{k-i}f(x+ih)
 \end{equation}
 Now prove (2.6) by an induction. For $k=1$, it could be verified directly:
 \begin{equation*}
  \binom{1}{0}(-1)^{1-0}f(x) + \binom{1}{1}(-1)^{1-1}f(x+h) = f(x+h)-f(x) = \Delta f(x)
 \end{equation*}
 Suppose (2.6) holds for some $k\geq 1$, then
 \begin{align*}
  \Delta^{k+1}f(x) &= \Delta\left(\sum_{i=0}^k \binom{k}{i}(-1)^{k-i}f(x+ih)\right)\\
  &= \sum_{i=0}^k \binom{k}{i}(-1)^{k-i}f(x+(i+1)h) - \sum_{i=0}^k \binom{k}{i}(-1)^{k-i}f(x+ih)\\
  &= f(x+(k+1)h) - (-1)^kf(x) + \sum_{i=1}^k \left(\binom{k}{i-1}(-1)^{k+1-i}f(x+ih)-\binom{k}{i}(-1)^{k-i}f(x+ih)\right)\\
  &= f(x+(k+1)h) + (-1)^{k+1}f(x) + \sum_{i=1}^k (-1)^{k+1-i}f(x+ih)\left(\binom{k}{i-1}+\binom{k}{i}\right)\\
  &= f(x+(k+1)h) + (-1)^{k+1}f(x) + \sum_{i=1}^k \binom{k+1}{i}(-1)^{k+1-i}f(x+ih)\\
  &= \sum_{i=0}^{k+1} \binom{k+1}{i}(-1)^{k+1-i}f(x+ih)
 \end{align*}
 It shows that (2.6) holds for $(k+1)$. Hence (2.4) is proved by induction. Now we prove that
 \begin{equation}
  \Delta^k f(x) = \nabla^k f(x+kh)
 \end{equation}
 by an induction. For $k=1$, it could be verified directly:
 \begin{equation*}
  \Delta f(x) = f(x+h)-f(x) = \nabla f(x+h)
 \end{equation*}
 Suppose (2.7) holds for some $k\geq 1$, then
 \begin{align*}
  \Delta^{k+1} f(x) &= \Delta\left(\Delta^k f(x)\right) = \Delta \left(\nabla^k f(x+kh)\right) = \nabla^k f(x+(k+1)h) - \nabla^k f(x+kh) \\
  &= \nabla\left(\nabla^k f(x+(k+1)h)\right) = \nabla^{k+1}f(x+(k+1)h)
 \end{align*}
 Hence (2.7) is proved by induction. Finally, (2.5) follows immediately from (2.4),(2.7) and Corollary 2.15.
\end{solution}

\vspace{1.5em}

\begin{problem}
  Assume $f$ is differentiable at $x_0$. Prove
  \begin{equation}
    \frac{\partial}{\partial x_0} f[x_0,x_1,...,x_n] = f[x_0,x_0,x_1,...,x_n]
  \end{equation}
  What about the partial derivate with respect to one of the other variables?
\end{problem}

\begin{solution}
  Firstly, follows from Definition 2.34, we have
  \begin{equation*}
    \frac{\partial}{\partial x_0}f[x_0]=f'(x_0)=f[x_0,x_0]
  \end{equation*}
  Prove (2.8) by an induction on $n$. For $n=1$, verify it directly:

  \begin{align*}
    \frac{\partial}{\partial x_0} f[x_0,x_1] &= \frac{\partial}{\partial x_0}\left(\frac{f[x_1]-f[x_0]}{x_1-x_0}\right)\\
    &= \frac{-(x_1-x_0)\frac{\partial}{\partial x_0}f[x_0]+f[x_1]-f[x_0]}{(x_1-x_0)^2}\\
    &= \frac{f[x_0,x_1]-f[x_0,x_0]}{x_1-x_0}\\
    & = f[x_0,x_0,x_1]
  \end{align*}
  Suppose (2.8) holds for some $n\geq 1$, then
  \begin{align*}
    \frac{\partial}{\partial x_0} f[x_0,x_1,...,x_{n+1}] &= \frac{\partial}{\partial x_0}\left(\frac{f[x_1,...,x_{n+1}]-f[x_0,...,x_{n}]}{x_{n+1}-x_0}\right)\\
    &= \frac{-(x_{n+1}-x_0)\frac{\partial}{\partial x_0}f[x_0,x_1,...,x_{n}]+f[x_1,...,x_{n+1}]-f[x_0,...,x_n]}{(x_{n+1}-x_0)^2}\\
    &= \frac{-(x_{n+1}-x_0)f[x_0,x_0,x_1,...,x_{n}]+f[x_1,...,x_{n+1}]-f[x_0,...,x_n]}{(x_{n+1}-x_0)^2}\\
    &= \frac{-f[x_0,x_0,x_1,...,x_{n}]+f[x_0,x_1,...,x_{n+1}]}{x_{n+1}-x_0}\\
    & = f[x_0,x_0,x_1,...,x_{n+1}]
  \end{align*}
  It shows that (2.8) holds for $(n+1)$, hence proved. Morever, the order of $x_0,...,x_n$ is not important, hence
  \begin{equation*}
    \frac{\partial}{\partial x_j} f[x_0,x_1,...,x_n] = f[x_0,...,x_{j-1},x_j,x_j,x_{j+1},...,x_n], \qquad \forall j=0,...,n
  \end{equation*}
\end{solution}

\vspace{1.5em}

\begin{problem}
  (A min-max problem) For $n\in\mathbb{N}^+$, determine
  \begin{equation}
    \min \max_{x\in[a,b]} |a_0x^n+a_1x^{n-1}+...+a_n|
  \end{equation}
  where $a_0\neq 0$ is fixed and the minimum is taken over all $a_i\in\mathbb{R},\;i=1,2,...,n$.
\end{problem}

\begin{solution}
  The map
  \begin{equation*}
    p(x) \mapsto q(x)=\frac{1}{a_0}p\left(a+\frac{b-a}{2}(x+1)\right)
  \end{equation*}
  yields a bisection relation between polynomials of degree $n$ defines in $[a,b]$ with leading coefficient $a_0$ and polynomials of degree $n$ defines in $[0,1]$ with leading coefficient $1$. Chebyshev's Theorem gives that
  \begin{equation*}
    \forall q\in \tilde{\mathbb{P}}_n, \qquad \max_{x\in[-1,1]} \left|\frac{T_n(x)}{2^{n-1}}\right|\leq \max_{x\in[-1,1]} |q(x)|
  \end{equation*}
  where $T_n$ is the Chebysheve's polynomial of oeder $n$. Hence the solution of the min-max problem $p_\text{min}(x)$ satisfies
  \begin{equation*}
    \frac{1}{a_0}p_\text{min}\left(a+\frac{b-a}{2}(x+1)\right)=\frac{T_n(x)}{2^{n-1}}
  \end{equation*}
  The result follows immediately:
  \begin{equation*}
    p_\text{min}(x)=\frac{a_0}{2^{n-1}}T_n\left(\frac{2}{b-a}(x-a)-1\right)
  \end{equation*}
  The min value in (2.8) is $\frac{|a_0|}{2^{n-1}}$.
\end{solution}

\begin{problem}
  (Imitate the proof of Chebyshev's Theorem) Express the Chebyshev polynomial of degree $n\in\mathbb{N}$ as a polynomial $T_n$ and change its domain from $[-1,1]$ to $\mathbb{R}$. For a fixed $a>1$, define $\mathbb{P}_n^a := \{p\in\mathbb{P}_n:p(a)=1\}$ and a polynomial $\hat{p}_n(x)\in\mathbb{P}_n^a$,
  \begin{equation*}
    \hat{p}_n(x) := \frac{T_n(x)}{T_n(a)}
  \end{equation*}
  Prove
  \begin{equation*}
    \forall p\in \mathbb{P}_n^a,\qquad ||\hat{p}_n||_\infty \leq ||p||_\infty
  \end{equation*}
  where the max-norm of a function $f:\mathbb{R}\to\mathbb{R}$ is defined as $||f||_\infty = \max_{x\in[-1,1]}|f(x)|$.
\end{problem}

\begin{solution}
  First we know that $||\hat{p}_n||_\infty=\frac{1}{|T_n(a)|}$. And by the property of $T_n$ we have
  \begin{equation*}
    \hat{p}_n(x'_k) = \frac{(-1)^k}{T_n(a)} \quad \text{for} \quad x'_k=\cos\frac{k}{n}\pi,\;k=0,1,...,n
  \end{equation*}
  Now we prove the conclution by using reduction to absurdity. Suppose that:
  \begin{equation*}
    \exists p\in \mathbb{P}_n^a,\quad \text{s.t.} \quad ||p||_\infty<\frac{1}{|T_n(a)|}
  \end{equation*}
  Let $q(x)=p(x)-\hat{p}_n(x)\in \mathbb{P}_n$, then $q(a)=0$. And
  \begin{equation*}
    q(x'_k) = p(x'_k) - \frac{(-1)^k}{T_n(a)},\quad k=0,1,...,n
  \end{equation*}
  We have $\text{sgn}(q(x'_k))\neq \text{sgn}(q(x'_{k-1}))$ for $k=1,...,n$ since $||p||_\infty<\frac{1}{|T_n(a)|}$. By the continuity of $q$,
  \begin{equation*}
    \exists -1=x_n<\xi_n<x_{n-1}<...<x_{1}<\xi_1<x_0=1, \quad \text{s.t.} \quad q(\xi_1)=\cdots =q(\xi_n)=0
  \end{equation*}
  However, $q(a)=0$ and $a>1$ shows that $q$ has at least $n+1$ zero points, that contradict to $q\in\mathbb{P}_n$.
\end{solution}

\vspace{1.5em}

\begin{problem}
  Prove Lemma 2.48:
  \begin{align}
    \forall k=0,1,...,n, \forall t\in(0,1),\quad b_{n,k}(t)>0 \\
    \sum_{k=0}^n b_{n,k}(t)=1\\
    \sum_{k=0}^n kb_{n,k}(t) = nt\\
    \sum_{k=0}^n (k-nt)^2b_{n,k}(t)=nt(1-t)
  \end{align}
  where
  \begin{equation*}
    b_{n,k}(t) = \binom{n}{k}t^k(1-t)^{n-k}
  \end{equation*}
\end{problem}

\begin{solution}
  (2.10) is clearly since $t\in(0,1)$.
  
  By the Binomial Theorem we have:
  \begin{equation*}
    1 = (t+(1-t))^n = \sum_{k=0}^n \binom{n}{k}t^k(1-t)^{n-k} = \sum_{k=0}^n b_{n,k}(t)
  \end{equation*}
  Hence (2.11) is proved.

  Again, by the Binomial Theorem we have:
  \begin{equation*}
    (p+q)^n=\sum_{k=0}^n \binom{n}{k}p^k q^{n-k}
  \end{equation*}
  Partial derivate with respect to $p$ to both sides yields:
  \begin{equation*}
    n(p+q)^{n-1}=\sum_{k=0}^n \binom{n}{k}kp^{k-1} q^{n-k}
  \end{equation*}
  Multiple a $p$ to both sides, yield
  \begin{equation}
    np(p+q)^{n-1}=\sum_{k=0}^n \binom{n}{k}kp^k q^{n-k}
  \end{equation}
  Now take $p=t$ and $q=1-t$, yield
  \begin{equation*}
    nt=\sum_{k=0}^n \binom{n}{k}kt^k (1-t)^{n-k} = \sum_{k=0}^n kb_{n,k}(t)
  \end{equation*}
  Hence (2.12) is proved.

  Follows from (2.14), partial derivate again with respect to $p$ to both sides yields:
  \begin{equation*}
    n(p+q)^{n-1} + n(n-1)p(p+q)^{n-2} =\sum_{k=0}^n \binom{n}{k}k^2p^{k-1} q^{n-k}
  \end{equation*}
  Multiple a $p$ to both sides, yield
  \begin{equation*}
    np(p+q)^{n-1} + n(n-1)p^2(p+q)^{n-2} =\sum_{k=0}^n \binom{n}{k}k^2p^k q^{n-k}
  \end{equation*}
  Now take $p=t$ and $q=1-t$, yield
  \begin{equation*}
    nt + n(n-1)t^2 = \sum_{k=0}^n k^2 b_{n,k}(t)
  \end{equation*}
  By (2.11),(2,12) and the result abouve, we got:
  \begin{align*}
    \sum_{k=0}^n (k-nt)^2b_{n,k}(t) &= \sum_{k=0}^n k^2b_{n,k}(t) - 2nt\sum_{k=0}^n kb_{n,k}(t) + (nt)^2\sum_{k=0}^n b_{n,k}(t)\\
    &= nt + n(n-1)t^2 - 2(nt)^2 + (nt)^2 = nt - nt^2 = nt(1-t)
  \end{align*}
  Hence (2.13) is proved.
\end{solution}

\end{document}